%Autor: Carlos de la Cruz (frodo@fmat.ull.es)

\chapter{PHP}

\section{Introducci�n}
PHP es un lenguaje dise�ado para la generaci�n din�mica de p�ginas web.
Los ficheros de PHP se almacenan en el servidor y se ejecutan cuando un usuario introduce en su navegador la direccion de una de estas p�ginas. Estas, a su vez, contienen c�digo a partir del cual se generan p�ginas est�ticas en formato HTML.

La utilidad de esto es que podemos crear autom�ticamente p�ginas web que se actualicen solas, bas�ndose en una fuente de datos.

Por ejemplo, si quisi�ramos que cada vez que el usuario accede a nuestra p�gina, en esta apareciera la fecha y la hora, ser�a imposible conseguirlo a partir de ficheros est�ticos ``html''.

Por tanto, la soluci�n ser�a tener en el servidor un fichero ``hora.php'', mediante el cual, cuando alguien accediera a: http://miservidor.com/hora.php, tuviera la hora del instante en que se ejecut� el c�digo php.

Cada vez que alguien le diera a ``recargar'' en el navegador, ver�a la hora actual.

Pero como podremos comprobar,si se hace clic en el bot�n ``ver c�digo fuente'' del navegador, s�lo veremos, por ejemplo ``12:02:12''.



\section{Primeros Pasos}

\section{Estructuras de datos}

B�sicamente, existen los escalares, los hash y los arrays.


\section{Estructuras de control}

\subsection{Condicionales: IF, Switch}


\subsection{Repetitivas: While}

\section{Como estructurar nuestro c�digo en php}

Las funciones. function chona($lalala,$otroparametro)

\section{Manejo de formularios}


\section{Utilizaci�n de bases de datos}
Aunque PHP soporta varias clases de servidores de base de datos, debido a que cada uno usa un interfaz distinto, resulta bastante inc�modo tener que aprender las funciones de PHP para cada uno de ellos. Es por ello que suelen utilizarse librer�as adicionales (no incluidas con PHP, pero s� con la mayor�a de las distribuciones Linux), como la que utilizaremos en este texto/curso: AdoDB.

AdoDB nos permite cambiar de sistema de bases de datos simplemente cambiando un par�metro al realizar la conexi�n a la Base de datos (BDD de ahora en adelante) , de forma que una aplicaci�n PHP dise�ada para correr con Oracle, puede ser portada para funcionar en MySql, Informix, etc. con tan s�lo cambiar una l�nea de c�digo.


\section{Creaci�n de im�genes din�micamente}

\section {Bibliograf�a recomendada}


