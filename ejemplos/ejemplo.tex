% Define el  tipo de documento  como "article" (artículo)  y especifica
% las opciones "a4paper" (tamaño de  papel A4), "12pt" (tamaño de letra
% a 12 puntos) y "twoside" (para imprimir a doble cara)
\documentclass[a4paper,12pt,twoside]{article}

% Utiliza  el paquete  "inputenc"  con la  opción  "latin1", esto  nos
% permite teclear las  "ñ" y las tildes sin tener  que preocuparnos de
% nada, porque el charset que estamos usando el latin1 y se lo decimos
% a LaTeX. Nota: Latin1 se refiere al estándar ISO-8859-1
\usepackage[latin1]{inputenc}

% Utiliza  el  paquete  "babel"  con   la  opción  "spanish",  lo  que
% incluye  entre  otras  cosas   traducción  de  elementos  insertados
% automáticamente, como "Índice general" en lugar de "Contents"
\usepackage[spanish]{babel}

% Utiliza el  paquete amssymb,  American Mathematical  Society SYMbols,
% (símbolos de  la Sociedad Matemática  Americana). Es este  paquete se
% encuentra  por ejemplo  el símbolo  de "isomorfo  a" utilizado  en el
% Teorema de Isomorfía.
\usepackage{amssymb}

% Utiliza el paquete  "eurosym" para proporcionar el  símbolo del euro.
% Con este  paquete podemos escribir el  símbolo del euro con  la orden
% \euro
\usepackage{eurosym}

% Ordenamos a LaTeX que no numere  las páginas, ya que tenemos una sóla
% página y un "1" solitario no queda muy bien.
\pagestyle{empty}

% Evita que  LaTeX introduzca  espacios mayores de  lo normal  tras los
% finales de las oraciones.
\frenchspacing

\begin{document} % Comienza el documento

% Esta frase estará centrada y con un tamaño de letra mayor
\begin{center}
{\large Taller de Iniciación a Linux para Alumnos}
\end{center}

Esto es un  pequeño ejemplo de {\LaTeX}, el más  potente procesador de
textos. La mayoría  de apuntes y exámenes de Matemáticas  que vemos en
la  Facultad están  escritos  en {\LaTeX}.  Por  cierto, con  {\LaTeX}
también estamos preparados para la llegada del {\euro}uro

A continuación algunos ejemplos de fórmulas matemáticas:

% Comenzamos una descripción de varios "items"
\begin{description}

\item [Definición de límite (Análisis Matemático I)]
$$
\lim_{x \longrightarrow a} = l \iff 
\forall \, \varepsilon > 0 \,
\, \exists \, \delta > 0 \, / \,
\, 0 < \| x - a \| < \delta \,
\Longrightarrow \, \| f(x) - l \| < \varepsilon
$$

\item [Teorema Generalizado de Cauchy (Análisis Matemático I)]
Si $f$ y $g$ tienen derivadas contínuas hasta el orden $n(n-1)$ en 
el intervalo $\lbrack a , b \rbrack$ y además $\forall \, x \in (a,b)
\, \, \exists \, f^{n)}(x), g^{n)}(x)$, entonces
$\exists \, c \in (a,b) \, /$
$$
\bigg( f(b) - \sum_{k=0}^{n-1} \frac{f^{k)}(a)}{k!}(b-a)^k \bigg) g^{n)}(x) =
f^{n)}(x) \bigg( g(b) - \sum_{k=0}^{n-1} \frac{g^{k)}(a)}{k!}(b-a)^k \bigg) 
$$

\item [$1^{er}$ Teorema de Isomorfía (Álgebra I)]
Sean $G$, $G'$ grupos, $f : G \longrightarrow G'$
homomorfismo de grupos. Entonces
$$
\frac{G}{Ker(G)} \, \thickapprox \, Im f
$$

\item [Funciones Eulerianas: Gamma y Beta (Análisis Matemático II)]
$$
\Gamma(p) = \int_{0}^{+\infty} e^{-x} x^{p-1} dx \quad \forall \, p > 0
$$
$$
\beta(p,q) = \int_{0}^{1} x^{p-1} (1-x)^{q-1} dx
$$

\item [Y por último, un ejemplo denso]
$$ \sum \limits_{n = 0}^{\infty} 
\left( \frac{\int \limits_{-\infty}^{+\infty}
{\left\lceil \frac{\sin \left[8 \frac{\pi}{3}^3 \right]}
{\arctan \left( \sqrt[3]{ 2 \cdot \sin {(x)} } \right)} \right\rceil dx} }
{\lim \limits_{x \to n^2} \left( \vert{ \frac{\log{\frac{\pi}{x^4}}}
{e^{\frac{n + 1}{n - 1}}} \vert} \right) } \right) =
\left| \begin{array}{cccc}
m_{(i,j)}   & m_{(i,j+1)}       & \ldots   & m_{(i,n)}     \\
m_{(i+1,j)} & m_{(i+1,j+1)}     & \ldots   & m_{(i+1,n)}   \\
\vdots        & \vdots              & \ddots & \vdots          \\
m_{(n,j)}   & m_{(i + n,j + 1)} & \ldots   & m_{(n,n)}
\end{array} \right| $$

\end{description}

\end{document} % Termina el documento
